The COBLL compiler is run primarily through {\ttfamily \mbox{\hyperlink{_c_o_b_l_l_8cpp}{COBLL.\+cpp}}}, which contains {\ttfamily main}. To use, the executable is run from the terminal with file names as command-\/line arguments.

A \mbox{\hyperlink{class_driver}{Driver}} (see {\ttfamily \mbox{\hyperlink{_driver_8h}{Driver.\+h}}}, {\ttfamily \mbox{\hyperlink{_driver_8cpp}{Driver.\+cpp}}}) is created for each file name supplied, and the file undergoes a first pass as it is constructed, in which the \mbox{\hyperlink{class_driver}{Driver}} locates each COBOL Division within the file. After it is constructed, control goes to {\ttfamily Driver\+Main}, which, for each division present within the file, calls that division\textquotesingle{}s handler. Division handlers have their own files, named {\ttfamily Identification\+Handler.\+cpp}, etc. They are all methods within the \mbox{\hyperlink{class_driver}{Driver}} so they can share data. The division handlers, since they are so different, are not part of a hierarchy that shares any functionality. For instance, the Identification\+Handler uses key-\/value pairs, while the Procedure Division relies on tokens as well as column information. 